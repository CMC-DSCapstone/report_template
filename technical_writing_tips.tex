\chapter{Technical Writing Tips}\label{Ch:technical_writing_tips}

\begin{enumerate}{}
\item Always remember who your audience is. Your readers include both your liason with technical knowledge of you methods to the company executives with a college level, general education knowledge of science.  
\item Always write with pedagogy in mind. You are trying to teach your reader what is it you did, what the techniques are based off of, and what they can be used for.  
\item Don't fall into the trap of trying to sound smart and display what you know. If you're using technical jargon unnecessarily, you will only confuse and frustrate your audience. 
\item Technical writing is succinct and to the point. Each sentence needs to be short, often single clauses and use accessible, non-flowery language. 
\item Avoid run-on sentences. The words "which" and "where" are your enemy here because they are often used to combine clauses and increase the word length of your sentences. Of course, having shorter sentences might increase the length of your document, but reading a lot of simple, readable text is far better than slowly slogging through dense, erudite text.
\item Avoid repetition. You might need to repeatedly refer to some important word in your document (think "data," or "system," or "model"). But if the reader keeps reading the same word over and over again in adjacent sentences, it starts to ring in their head and the writing comes off amateurish. Reword to avoid repetition.
\item Citations, citations, citations; do not forget to cite relevant work or facts you mention. The rule of thumb is that if you say a fact that is both non-obvious and for which you personally did not perform the experiment to demonstrate, then you must cite something.
\item The usual English writing rules still hold: avoid ending your sentences in prepositions; avoid passive voice; don't start sentences with numerals; etc.
\end{enumerate}


\endinput