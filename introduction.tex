\chapter{Introduction}\label{Ch:Introduction}

The introduction should describe what the purpose of the project is/was and what you have accomplished.
The introduction, as well as other parts of the report, can be developed incrementally and will evolve with the project.
The introduction should contain the following items:
\begin{enumerate}
\item a brief description of the sponsoring organization (which often can be derived from the organization's online boilerplate), 
\item a suitably condensed statement of the problem posed by the sponsor, 
\item some discussion of the relevance of the project to the sponsor's business, 
\item the team's approach to the problem, gleaned from the team's Statement of Work (including a summary of background study, i.e., literature review, to explain how your work differs from or builds on previous efforts --- this explanation should be reinforced by entries with annotation in the bibliography),
\item summary of the report: in separate paragraphs specify and summarize the major sections of your report (Ch 2, Ch 3, ... , Conclusion,  Appendixes, Glossary, Bibliography).
\end{enumerate}

\begin{quote}
    \S{} \LaTeX{} Tip: When you create your report, we recommend you to create different TeX file for different chapter. In this example, you will also find several LaTeX files comprising multiple chapters as the main body.  
\end{quote}



\endinput