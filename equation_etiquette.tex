\chapter{Equation Etiquette}\label{Ch:equation_etiquette}

This is an example of an equation that comes at the end of a sentence:
%
\begin{equation}
    \mathrm{Area} = \pi r^2.
\end{equation}
%
Note the above has a period at the end of the equation because it was at the end, but the following equation,
%
\begin{equation}
    p_{\mathrm{Normal}}(x\vert \mu, \sigma^2) = \frac{1}{\sqrt{2 \pi \sigma^2}}\mathrm{exp}\left[-\frac{(x - \mu)^2}{2 \sigma^2}\right],
\end{equation}
%
has a comma at the end because it's in the middle of a sentence.

Here's an example of a multi line equation using the align command:
%
\begin{align}
    \mathrm{E}[X] &= \sum_{n=1}^{6} n \mathrm{Pr}(X=n)\nonumber\\
    &= \sum_{n=1}^{6} n \frac{1}{6}\nonumber\\
    &= 3.6.
\end{align}
%
Take note of the use of the nonumber command to suppress the extra equation numbers.


\begin{enumerate}{}
\item Use equation for single line equations and use align for multiple line equations.
\item For multiple line equations, use nonumber to suppress the extra equation numbers.
\item Math symbols and variables are italicized; English words are not. Thus, use mathrm when in Mathmode to specify the English words. This is a subtle point, but subscripts and function names that are actually abbreviations of English words should still be treated as normal English.
\item Surround your equation/align commands by \% so that there are no unintentional new paragraphs while making your Latex more readable.
\item Math equations are part of your sentence, so use punctuation accordingly: if the equation is at the end, use a period; if the equation is in the middle of your sentence, use a comma.
\item Create labels for your equations and when you reference them, use the tilda to create a small space as so: You can find the result in Eq.\textasciitilde\textbackslash ref\{EQ\_IMPORTANTRESULT\}.
\item For inline math, surround your math with dollar signs. Keep in mind the format might look a little different to fit large symbols, such as in $\sum_{n=1}^{6} \frac{1}{6} = 1$.
\end{enumerate}


\endinput